% !Mode:: "TeX:UTF-8"

数字图像中处理的是离散的值,对于一维函数的一阶
微分的基本定义是差值:\\ 
\[\frac{\partial f}{\partial x}=f\left(x+1\right)-f\left(x\right)\]
\\类似的,可将二阶微分定义为:\\
\[\frac{\partial^2f}{\partial x^2}=f\left(x+1\right)+f\left(x-1\right)-2f(x)\]
\\将如上一维函数扩展到二维,可得:\\
\[\frac{\partial^2f}{\partial x^2}=f\left(x+1,\;y\right)+f\left(x-1,\;y\right)-2f(x,\;y)\]
\[\frac{\partial^2f}{\partial y^2}=f\left(x,\;y+1\right)+f\left(x,\;y-1\right)-2f(x,\;y)\]
\\二阶微分的定义保证了以下几点:
\begin{enumerate}
    \item 在恒定灰度区域的微分值等于0
    \item 在灰度台阶或斜坡的起点处微分值不为0
\end{enumerate}
所以,二阶微分可以检测出图像的边缘、增强细节\\
Laplace是最简单的各向同性微分算子,其滤波器的响应
与滤波器作用的图像的突变方向无关。\\
一个二维图像函数$f(x, y)$定义为:\\
\[\nabla^2f=\frac{\partial^2f}{\partial x^2}+\frac{\partial^2f}{\partial y^2}=f\left(x+1,\;y\right)+f\left(x-1,\;y\right)+f(x,y+1)+f(x,\;y-1)-4f(x,\;y)\]
\\实现上式的滤波器模板为:\\
0   1   0\\
1  -4   0\\
0   1   0\\
   